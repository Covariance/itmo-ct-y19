\section{}
	Пойдём от противного "--- предположим, что у $a$ и $a^r$ наибольшая общая подпоследовательность $s$, не равная по размеру наибольшему подпалиндрому $p$ строки $a$. Тогда есть два варианта:
	\begin{itemize}
		\item $|s| < |p|$. На самом деле такого быть не может, так как если развернуть $p$, то он останется самим собой, а значит, будет являться общей подпоследовательностью $a$ и $a^r$, значит, её длина не меньше $|p|$.
		\item $|s| > |p|$. Значит, есть какая-то подпоследовательность в $a$, такая, что она же есть в $a^r$. хммммм кажецца мы нашли подпалиндром. А если точнее, то это значит, что существует такие две последовательности индексов $i_1, i_2, \dotsc, i_|s|$ и $j_1, j_2, \dotsc, j_|s|$, что $\forall k:\:a_{i_k} = a^r_{j_k}$. Тогда рассмотрим минимальное $l$ такое, что $i_{l + 1} \geq |a| - j_{l + 1} + 1$. Заметим, что если в строке $a$ рассмотреть последовательность с индексами $i_1, _2, \dotsc, i_l, |a| - j_l + 1, |a| - j_{l-1} + 1, \dotsc, |a| - j_1 + 1$, то она очевидно будет определять подпалиндром строки $a$. По аналогичным причинам, последовательность $|a| - j_|s| + 1, |a| = j_{|s| - 1}, \dotsc, |a| - j_{l + 1} + 1, i_{l + 1}, i_l, \dotsc, i_{|s|}$ тоже определяет подпалиндром. Заметим, что хоть одна из них содержит в себе не меньше $|s|$ членов. А по предыдущей части доказательства их оказывается ровно $|s|$. 
	\end{itemize}
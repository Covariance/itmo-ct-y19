\documentclass{article}

\usepackage[T2A]{fontenc}
\usepackage[utf8]{inputenc}
\usepackage[russian]{babel}
\parindent 0pt
\parskip 8pt


%большие-пребольшие отступы
%\usepackage{setspace}

\usepackage{etaremune}
\usepackage{amsmath}
\usepackage{amssymb}
\usepackage{mathrsfs}
\usepackage{amsfonts}
\usepackage{amsthm}
\usepackage[left=2.3cm, right=2.3cm, top=2.7cm, bottom=2.7cm, bindingoffset=0cm]{geometry}
\usepackage{latexsym}
\usepackage[unicode, pdftex]{hyperref}
\usepackage{xcolor}%Для цветов текста и бэка
\usepackage{graphicx}
\usepackage{mathtools}
\usepackage{hyperref}%Ссылки
\usepackage{listings}%код
\graphicspath{ {./images/} }%Путь по умолчанию к картинкам

\everymath{\displaystyle}


%Задефаем цвета для отображения кода 

%с++
\lstset{
	backgroundcolor=\color{gray!10},  
	basicstyle=\ttfamily,
	columns=fullflexible,
	breakatwhitespace=false,      
	breaklines=true,                
	captionpos=b,                    
	commentstyle=\color{green!60!blue}, 
	extendedchars=true,              
	frame=single,                   
	keepspaces=true,             
	keywordstyle=\color{blue},      
	language=c++,                 
	numbers=none,                
	numbersep=5pt,                   
	numberstyle=\tiny\color{blue}, 
	rulecolor=\color{black!30},        
	showspaces=false,               
	showtabs=false,                 
	stepnumber=5,                  
	stringstyle=\color{red!60!blue},    
	tabsize=4,                      
	title=\lstname          
}

%python



\def\Z{\mathbb Z}
\def\R{\mathbb R}
\def\N{\mathbb N}
\def\Q{\mathbb Q}
\def\kratno{\mathrel{\smash{\lower.1ex\hbox{$\,\vdots\,$}}}}
\def\nekratno{\mathrel{\mathpalette\c@ncel\kratno}}
\def\c@ncel#1#2{\m@th\ooalign{$\hfil#1\mkern1mu/\hfil$\crcr$#1#2$}}
\def\q#1. {\noindent\phantom{1}{\bf#1.} }


\newtheorem{theorem}{Теорема}[section]
\newtheorem{corollary}{Следствие}[theorem]
\newtheorem{definition}{Определение}[section]
\newtheorem{problem}{Задача}[section]

\newtheorem{remark}[theorem]{Замечание}
\newtheorem{lemma}[theorem]{Лемма}
\newtheorem{example}[theorem]{Пример}
\newtheorem{proposition}[theorem]{Пояснение}

\title{АиСД\_дз4}
\author{CovarianceMomentum}
\date{1term}

\begin{document}
	
	\maketitle
	
	\section{}
	Пойдём от противного. Пусть всего в сортирующей сети $n$ нитей. Рассмотрим ситуацию, когда на входах сначала $i - 1$ ноль, затем единица, затем ноль, оставшиеся чила "--- нули. Получаем что-то вида $00000010111111$. Таким образом, ни одна из единиц, находящихся на позициях $[i +2; n]$ не может встать на позицию $i + 1$, так как им для этого придётся "подняться по сети", а единица с позиции $i$ не может встать на неё, так как её нить никогда не сравнивается с нитью $i+1$, а подняться выше она не может.\newpage
	
	\section{}
	Пойдём от противного "--- предположим, что у $a$ и $a^r$ наибольшая общая подпоследовательность $s$, не равная по размеру наибольшему подпалиндрому $p$ строки $a$. Тогда есть два варианта:
	\begin{itemize}
		\item $|s| < |p|$. На самом деле такого быть не может, так как если развернуть $p$, то он останется самим собой, а значит, будет являться общей подпоследовательностью $a$ и $a^r$, значит, её длина не меньше $|p|$.
		\item $|s| > |p|$. Значит, есть какая-то подпоследовательность в $a$, такая, что она же есть в $a^r$. хммммм кажецца мы нашли подпалиндром. А если точнее, то это значит, что существует такие две последовательности индексов $i_1, i_2, \dotsc, i_|s|$ и $j_1, j_2, \dotsc, j_|s|$, что $\forall k:\:a_{i_k} = a^r_{j_k}$. Тогда рассмотрим минимальное $l$ такое, что $i_{l + 1} \geq |a| - j_{l + 1} + 1$. Заметим, что если в строке $a$ рассмотреть последовательность с индексами $i_1, _2, \dotsc, i_l, |a| - j_l + 1, |a| - j_{l-1} + 1, \dotsc, |a| - j_1 + 1$, то она очевидно будет определять подпалиндром строки $a$. По аналогичным причинам, последовательность $|a| - j_|s| + 1, |a| = j_{|s| - 1}, \dotsc, |a| - j_{l + 1} + 1, i_{l + 1}, i_l, \dotsc, i_{|s|}$ тоже определяет подпалиндром. Заметим, что хоть одна из них содержит в себе не меньше $|s|$ членов. А по предыдущей части доказательства их оказывается ровно $|s|$. 
	\end{itemize}\newpage
	
	\section{}
	Идея решения такова "--- вместе с оригинальным ( $\operatorname{original}$) массивом инициализируем два массива такой же длины "---  $\operatorname{iter}$ и  $\operatorname{filled}$ и будем поддерживать количество уже инициализированных элементов  $\operatorname{cnt}$. Когда мы делаем  $\operatorname{set}$ какого-либо элемента  $\operatorname{original}$, который мы ещё не трогали, в ячейку с таким же номером в  $\operatorname{iter}$ кладём  $\operatorname{cnt}$, а в ячейку  $\operatorname{filled}$ с номером  $\operatorname{cnt}$ кладём позицию самого элемента и увеличиваем  $\operatorname{cnt}$.
	
	Теперь посмотрим, как мы будем проверять, инициализирован ли элемент на позиции  $i$. Если да, то ячейка массива $\operatorname{iter}$, на которую указывает  $\operatorname{filled}[i]$, лежит в пределе уже заполненных и должна указывать на  $i$. Почему такого не может случиться, если элемент неинициализирован, очевидно "--- первые  $\operatorname{cnt}$ элементов массива  $\operatorname{filled}$ указывают только на уже инициализированные элементы. Ниже приведён код, реализующий эту структуру данных.
	
	\lstinputlisting[language=c++]{3.cpp}
	
\end{document}
\section{}
	НУО $n < m$. Пусть меньший массив называется $a[n]$, больший "--- $b[m]$. Начнем связывать факты. Пусть в слиянии наших массивов до $k$-ой позиции стоит префикс из $i$ элементов $a[]$ и префикс из $k - i$ элементов $b[]$. Теперь поймём, может ли такой префикс \textbf{на самом деле} быть в слиянии. Это будет означать, что выполнено одно из двух условий:
	
	\begin{itemize}
		\item $b_{k-i} \le a_i \le b_{k-i+1}$
		\item $a_i \le b_{k-i} \le a_{i+1}$
	\end{itemize}

	Чтобы не заморачиваться с концами массива, будем считать $a[n]$ и $b[m]$ максимальным значением нашего типа данных. Если таких нет, ну, придётся ручками рассматривать, но это мелочи.
	
	Почему нужно, чтобы выполнялись именно эти условия? Ну, они просто означают, что именно конкретный элемент (в первом случае $a_i$, во втором "--- $b_{k-i}$) корректно завершает префикс слияния. 
	
	Окей, теперь мы умеем проверять, подходит нам такое $i$ или нет. Теперь, в случае если оно не подходит, научимся понимать, меньше или больше настоящее $i$, чем текущее. Опять же, проверяем несколько условий:
	
	\begin{itemize}
		\item Если $a_{i+1} \le b_{k-i}$, то $i$ слишком маленькое.
		\item Если $b_{k-i+1} \le a_i$, то $i$ слишком большое.
	\end{itemize}

	Опять же, понятно, почему условия именно такие "--- если последний элемент префикса массива $a[]$ слишком большой, то надо взять префикс поменьше, а если последний элемент префикса массива $b[]$ слишком большой, то надо взять префикс $a[]$ побольше.
	
	Теперь запустим бинпоиск по массиву $a[]$, который будет искать такой номер $i$, что соответствующие ему префиксы легитимно сливаются. Условие вызода указано выше, правило выбора, в какую половину рабочего отрезка переходить "--- тоже. Бинпоиск работает за $\mathcal{O}(\log n)$, а так как мы приняли $n < m$, то это именно то, что нам нужно.
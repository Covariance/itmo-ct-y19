\section{}
	Положим, что количество блоков, данных нам, равно $N$. Разобьём процесс запросов памяти на два этапа "--- до того, как мы выдали суммарно $N$ блоков памяти и после. До того, как мы выдали $N$ кусков памяти, будем просто выдавать их по порядку. После этого будем выдавать их из очереди, которая будет формироваться в процессе возвращения нам блоков памяти. Ниже приведен код, схематически показывающий этот процесс.
	
	Так как операции добавления и изъятия из очереди выполняются за $\mathcal{O}(1)$, то и весь алгоритм будет выполняться за $\mathcal{O}(1)$.
	
	\lstinputlisting[language=c++]{4.cpp}
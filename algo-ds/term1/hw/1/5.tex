\section{}
\label{sec:5}
	\subsection{Первый алгоритм}
		Будем считать одним входным значением алгоритма половину максимальную длину аргумента.
		
		Тогда довольно несложно заметить, что рекуррентную ассимптотику работы алгоритма можно оценить как $T(n) \leq 4T(\frac{n}{2}) + 13n$.
		
		По мастер-теореме, так как $1 < \log_24$, то $T(n) = \mathcal{O}(n^2)$.
		
	\subsection{Второй алгоритм} 
		Будем считать одним входным значением алгоритма половину максимальную длину аргумента.
		
		Тогда довольно несложно заметить, что рекуррентную ассимптотику работы алгоритма можно оценить как $T(n) \leq 3T(\frac{n}{2}) + 13n$.
		
		По мастер-теореме, так как $1 < \log_23$, то $T(n) = \mathcal{O}(n^{\log_23})$.
\section{}
	Для начала заметим один очень важный факт "--- любое двоичное дерево высоты $n + 1$ можно получить из какого-либо двоичного дерева высоты $n$ при помощи операции "раздвоения" и "продления" вершины . Покажем от противного "--- пусть это не так и существует дерево, не получаемое такими , тогда "сольём" все вершины c глубиной $n + 1$ в их родителей, и получим двоичное дерево с максимальной глубиной $n$, чтд. 

	Теперь докажем по индукции по глубине дерева.
	
	\textbf{База:} для $n = 0$ очевидно, так как это дерево либо пустое, либо в нём есть ровно одна вершина, глубина которой равна $0$. Таким образом, в первом случае указанная сумма равна нулю, во втором же равна в точности единице.
	
	\textbf{Переход:} пусть для глубин меньше $n$ доказано. Докажем для произвольного дерева $T_1$ с максимальной глубиной $n + 1$. Для этого рассмотрим дерево $T_0$ с максимальной глубиной $n$, из которого получается наше дерево. Для каждой вершины в $T_0$, у которой есть дети в $T_1$, заметим, что она сама теперь не участвует в сумме, зато её дети участвуют. Таким образом получаем, что в новом дереве сумма вершин не увеличится, так как даже если у этой вершины двое детей, то $2^{-n} = 2^{-n-1} + 2^{-n-1}$, а если один, то $2^{-n} > 2^{-n-1}$. Таким образом, при увеличении максимальной глубины указанная сумма для дерева не увеличивается, чтд.
	
	Критерий для равенства указанной суммы единице прост "--- в дереве у каждой нелистовой вершины два ребёнка. Тогда в предыдущем доказательстве всегда соблюдается равенство при переходе на следующую глубину, а начальная сумма равна единице.
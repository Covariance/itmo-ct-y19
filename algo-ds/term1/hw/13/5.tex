\section{}
	Заметим следующий факт:
	
	\begin{equation}
		\sum_{i = 1}^{n} 2^i \cdot (n - i + 1) = 2^{n + 2} - 2n - 4
	\end{equation}
	
	Доказывается по индукции:
	\textbf{База:} при $n = 1$ получаем $2 \cdot (1 - 1 + 1) = 8 - 2 - 5$, что верно.
	
	\textbf{Переход:} пусть для $n$ верно. Перейдем к $n+1$. В левой части в каждом слагаемом добавится один в скобочку, а ещё добавится $2 ^ {n + 1}$. Итого сумма добавленного будет равна $\sum_{i = 1}^{n+1} 2^i = 2^{n + 2} - 2$. Что закономерно, в правой части добавится это же значение. Победа.

	Рассмотрим то, как лежит наше множество из $k$ вершин. Утверждается, что оно лежит в двух поддеревьях высотой не более, чем $\lceil\log_2k\rceil - 1$ и, возможно, задевает одну высокую вершину между этими поддеревьями (одно из поддеревьев может быть пустым). Рассмотрим поддерево высотой $\lceil\log_2k\rceil - 1$, в котором лежит стартовая вершина этого множества. Есть два варианта: если все множество лежит в этом поддереве и если переходит через высокую вершину. Оно не может перейти через высокую вершину больше одного раза потому, что тогда ему придется пройти полностью поддерево размером $\lceil\log_2k\rceil - 1$ и еще хотя бы две вершины, а это уже $k + 1$ посещенная вершина. 
	
	Поймём, что для обход поддерева высотой $\lceil\log_2k\rceil - 1$ занимает $\mathcal{O}(k)$ переходов по ребрам. Рассмотрим вершину. Есть путь в неё и из неё, оба длиной $h$. Рассмотрим её детей. Для каждого из них суммарная длина путей в них и из них будет $2\cdot (h - 1)$, суммарно $4\cdot (h - 2)$. Получается сумма из первого равенства. Таким образом суммарное количество переходов по рёбрам будет не больше $2^{\lceil\log_2k\rceil + 1} \leq 4 \cdot k$. Проход же по высокому разделителю точно будет выполнен за $2 \cdot \log_2n$ переходов по рёбрам. Таким образом получаем, что суммарное количество переходов будет не больше, чем $8\cdot k + 2 \cdot \log_2n = \mathcal{O}(k + \log_2n)$.
	
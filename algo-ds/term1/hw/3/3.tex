\section{}
	Вспомним, как мы строили перестановки из задачи 1:
	\begin{enumerate}
		\item Ставим элемент $x_1\neq 1$ на первую позицию;
		\item Переходим в позицию $x_1$;
		\item Ставим элемент $x_2 \neq x_1 \neq 1$ на позицию $x_1$;
		\item Переходим в позицию $x_2$;
		\item $\dotsc$
		\item Переходим в позицию $x_n$;
		\item Ставим в неё $1$. 
	\end{enumerate}

	Теперь вспомним, как мы строили перестановки из задачи 2:
	\begin{enumerate}
		\item Ставим элемент $x_1 \neq 1$ на первую позицию;
		\item Переходим в позицию $2$;
		\item Ставим элемент $x_2 \neq x_1 \neq 1$ на вторую позицию;
		\item Переходим в позицию $3$;
		\item $\dotsc$
		\item Переходим в позицию $n$;
		\item Ставим в неё $1$.
	\end{enumerate}

	Схожесть нетрудно заметить. Теперь приведем способ перейти из какой-то определённой перестановки первого вида в перестановку второго обратно.
	
	$1 \rightarrow 2$\\
	Попросту выпишем все $x_j$ в ряд и получим перестановку второго рода.
	
	$2 \rightarrow 1$\\
	Пойдём по нашей перестановке, создавая перестановку первого рода по алгоритму, каждый раз выбирая очередной член нашей последовательности как $x_j$, получим перестановку второго рода.
	
	Биективная перестановка вычисляется за $\mathcal{O}(n)$ так как для её вычисления надо всего лишь раз пройтись по оригинальной перестановке.
\section{}
	Предлагается следующий алгоритм:
	\begin{enumerate}
		\item Разбиваем оригинальный массив на блоки по $k$ элементов;
		\item Сортируем подмассив, образованный первым и вторым блоками;
		\item Сортируем подмассив, образованный вторым и третьим блоками;
		\item $\dotsc$
		\item Сортируем подмассив, образованный предпоследним и последним блоками.
	\end{enumerate}

	Докажем, что это работает. Инвариантом будет то, что элементы в предыдущих блоках уже стоят на своих местах. Тогда если мы сортируем новую пару блоков, то все элементы, которые должны были стоять в первом блоке этой пары точно содержатся в паре, так как они не могут быть в предыдущих блоках, потому как те уже отсортированы. Значит, эти элементы встанут на свои места "--- мы отсортировали ещё один блок, отсортированный префикс увеличился, переходим дальше.
	
	Имеется $\frac{n}{k}$ блоков, каждую пару из которых мы сортируем за $\mathcal{O}(k\log k)$. Таким образом, итоговая асимптотика равна $2\cdot\frac{n}{k}\cdot\mathcal{O}(k\log k) = \mathcal{O}(\frac{n}{k}\cdot k\log k) = \mathcal{O}(n\log k)$.
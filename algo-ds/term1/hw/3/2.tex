\section{}
	Покажем, что в любой перестановке, где $1$-ый элемент стоит на последнем месте, сортировка пузырьком совершит все итерации. Заметим, что на каждой итерации $1$ сдвигается только на одну позицию, так как не существует элементов меньше неё, а значит за одну итерацию её можно подвинуть только один раз. Таким образом показали, что если единица стоит на последней позиции, то пузырёк делает максимальное число итераций.
	
	Теперь докажем, что другие перестановки завершаются за меньшее число итераций. Пусть $1$ стоит на позиции $x$. Тогда через $x-1$ итерацию первый элемент встанет на своё место, а суффикс массива длиной $x-1$ будет отсортирован. В оставшейся и, возможно, неотсортированной части массива будет $n - x$ элементов, которые будут отсортированы за максимум $n-x-1$ итерацию пузырька. Таким образом, если только $x\neq 0$, пузырёк отсортирует массив за не больше чем $(x-1) + (n-x-1) = x-2$ итераций пузырька. Победа. 
	
	Также поймём, что интересующих нас перестановок всего $(n-1)!$, так как помимо единицы на последнем месте мы можем разместить элементы в любом порядке.
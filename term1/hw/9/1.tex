\section{}
	Храним массивы:
	\begin{itemize}
		\item $\operatorname{arr}[n]$, собственно, наша последовательность.
		\item $\operatorname{lastPosition}[n]$, для которого $\operatorname{lastPosition}[i]$ это индекс последнего вхождения элемента $i$ в последовательность. Изначально все его члены равны $-1$;
		\item $\operatorname{seq}[n]$, который будет пересчитываться динамически. В элементе $\operatorname{seq}[i]$ хранится количество последовательностей, заканчивающихся в этом элементе.
		\item $\operatorname{pref}[n]$, который будет пересчитываться динамически. В элементе $\operatorname{pref}[i]$ хранится $\sum_{k = 0}^i seq[k]$.
	\end{itemize}   

	Опишем пересчёт динамики при переходе $i \rightarrow i + 1$. Посмотрим, что лежит в $\operatorname{lastPosition}\left[\operatorname{arr}[j + 1] - 1\right]$:
	\begin{itemize}
		\item Если там $-1$, значит, такое число никогда ещё не встречалось, следовательно, к любой из существующих последовательностей можно дописать его в конец, а также взять его само как отдельную последовательность из одного символа "--- присваиваем $\operatorname{seq}[j + 1]$ значение $\operatorname{pref}[j] + 1$.
		\item Если там индекс $k$, значит, такое число уже встречалось, следовательно, ко всем последовательностям, которые мы посчитали ДО его предыдущего вхождения, мы уже приписывали это число. Однако к последовательностям, которые заканчивались в элементах с индексами больше или равными $k$ и меньшими $j + 1$, мы не приписывали этого числа. Их количество будет равно $\operatorname{pref}[j] - \operatorname{pref}[k - 1]$. Присвоим его $\operatorname{seq}[j + 1]$.
	\end{itemize}
	
	После подсчёта $\operatorname{seq}[j + 1]$ посчитаем $\operatorname{pref}[j + 1] = \operatorname{pref}[j] + \operatorname{seq}[j + 1]$. Таким образом, переместились в следующее положение динамики. Ответом будет значение в $\operatorname{pref}[n - 1]$. 
	
	Так как алгоритм делает всего $\mathcal{O}(n)$ шагов, то и арифметических и других действий тоже будет $\mathcal{O}(n)$.
\documentclass{article}

\usepackage[T2A]{fontenc}
\usepackage[utf8]{inputenc}
\usepackage[russian]{babel}
\parindent 0pt
\parskip 8pt


%большие-пребольшие отступы
%\usepackage{setspace}

\usepackage{etaremune}
\usepackage{amsmath}
\usepackage{amssymb}
\usepackage{mathrsfs}
\usepackage{amsfonts}
\usepackage{amsthm}
\usepackage[left=2.3cm, right=2.3cm, top=2.7cm, bottom=2.7cm, bindingoffset=0cm]{geometry}
\usepackage{latexsym}
\usepackage[unicode, pdftex]{hyperref}
\usepackage{xcolor}%Для цветов текста и бэка
\usepackage{graphicx}
\usepackage{mathtools}
\usepackage{hyperref}%Ссылки
\usepackage{listings}%код
\graphicspath{ {./images/} }%Путь по умолчанию к картинкам

\everymath{\displaystyle}


%Задефаем цвета для отображения кода 

%с++
\lstset{
	backgroundcolor=\color{gray!10},  
	basicstyle=\ttfamily,
	columns=fullflexible,
	breakatwhitespace=false,      
	breaklines=true,                
	captionpos=b,                    
	commentstyle=\color{green!60!blue}, 
	extendedchars=true,              
	frame=single,                   
	keepspaces=true,             
	keywordstyle=\color{blue},      
	language=c++,                 
	numbers=none,                
	numbersep=5pt,                   
	numberstyle=\tiny\color{blue}, 
	rulecolor=\color{black!30},        
	showspaces=false,               
	showtabs=false,                 
	stepnumber=5,                  
	stringstyle=\color{red!60!blue},    
	tabsize=4,                      
	title=\lstname          
}

%python



\def\Z{\mathbb Z}
\def\R{\mathbb R}
\def\N{\mathbb N}
\def\Q{\mathbb Q}
\def\kratno{\mathrel{\smash{\lower.1ex\hbox{$\,\vdots\,$}}}}
\def\nekratno{\mathrel{\mathpalette\c@ncel\kratno}}
\def\c@ncel#1#2{\m@th\ooalign{$\hfil#1\mkern1mu/\hfil$\crcr$#1#2$}}
\def\q#1. {\noindent\phantom{1}{\bf#1.} }


\newtheorem{theorem}{Теорема}[section]
\newtheorem{corollary}{Следствие}[theorem]
\newtheorem{definition}{Определение}[section]
\newtheorem{problem}{Задача}[section]

\newtheorem{remark}[theorem]{Замечание}
\newtheorem{lemma}[theorem]{Лемма}
\newtheorem{example}[theorem]{Пример}
\newtheorem{proposition}[theorem]{Пояснение}

\title{АиСД\_дз9}
\author{CovarianceMomentum}
\date{1term}

\begin{document}
	
	\maketitle
	
	\section{}
	\begin{lstlisting}[language=Python]
	increment():
		i = 0
		while i < k and a[i] == 1:
			a[i] = 0
			i++
		if i < k:
			a[i] = 1
	\end{lstlisting}
	
	\begin{lstlisting}[language=Python]
	decrement():
		i = 0
		while i < k and a[i] == 0:
			a[i] = 1
			i++
		if i < k:
			a[i] = 0
	\end{lstlisting}
	
	\begin{lstlisting}[language=Python]
	get(i):
		return a[i]
	\end{lstlisting}
	
	\begin{lstlisting}[language=Python]
	setZero():
		i = 0
		while i < k:
			a[i] = 0
			i++
	\end{lstlisting}
	
	Предварительно положим, что за единицу времени выполняются операции сравнения, присваивания и арифметические действия.
	
	\paragraph{a)}
	Как нетрудно заметить, если весь массив $\operatorname{a}[]$ заполнен единицами, то истинное время работы $\operatorname{increment}$ будет равняться $1 + 4\cdot k$. Поймём, что в случае, если не весь массив заполнен единицами, а только первые $m < k$ позиций, то время работы будет равно $1 + 4\cdot m + 2 < 1 + 4\cdot k$. 
	
	\paragraph{b)}
	
	\begin{lemma}
		В арифметических преобразованиях в этой задаче мы будем пользоваться следующим фактом:
		$$\sum_{j = 0}^{n} j\cdot 2^j \;=\; 2^{n+1}\cdot (n-1) + 2$$
	\end{lemma}
	
	Поймём, что вариантов, когда первые $m < k$ элементов массива заполнены единицами и $\operatorname{a}[m] = 0$ будет ровно $2^{k-m-1}$, так как способов заполнить каждую из оставшихся позиций ровно $2$. Таким образом, суммарное время работы $\operatorname{increment}$ на всех возможных входных данных будет равняться:
	
	\begin{gather*}
	\left(\sum_{j = 0}^{k-1} 2^{k-j-1} \cdot (3 + 4\cdot j)\right) + 1+4\cdot k \,=\\
	=\, 3 \cdot (2^k - 1) + 4\cdot \sum_{i = 0}^{k-1} (2^{i} \cdot ((k-1)-i)) + (1 + 4\cdot k) \,=\\
	=\, (3 + 4\cdot k - 4) \cdot(2^k - 1) - 4\cdot \sum_{i = 0}^{k-1}  2^i\cdot i +  (1 + 4\cdot k) \,=\\
	=\, (4\cdot k - 1) \cdot(2^k - 1) - 4\cdot(2^k\cdot(k-2) + 2) + (1+4\cdot k) \,=\\
	=\, 7\cdot 2^k - 6
	\end{gather*}
	
	Так как всего вариантов входных данных $2^k$, то среднее время работы $\operatorname{increment}$ равняется $7 - \frac{6}{2^k} = 7$.
	
	\paragraph{c)}
	
	Рассмотрим массив, весь заполненный единицами, кроме последнего элемента. Будем поочередно применять к нему операции $\operatorname{increment}$ и $\operatorname{decrement}$. Тогда массив будет менять состояния между вышеописанным и всеми нулями с последним элементом единицей. $\operatorname{increment}$ и $\operatorname{decrement}$ в таком случае будут работать за $1 + 4\cdot k$. Теперь очевидно, что $n$ операций будут выполняться за $\Omega(nk)$. Осталось показать, что они не могут выполняться за большее время. Поймём, что время выполнения $\operatorname{decrement}$ на массиве $\operatorname{a}[]$ равно времени выполнения $\operatorname{increment}$ на инвертированном массиве (каждый элемент $\operatorname{a}[i] = 1 - \operatorname{a}[i]$), таким образом получаем, что $\operatorname{decrement}$ в худшем случае не может работать дольше, чем $\operatorname{increment}$ в худшем случае. Таким образом, показали, что рассмотренный случай действительно наихудший, $\Rightarrow$ время выполнения этих операций в худшем случае равно $\Theta(nk)$.
	
	\paragraph{d)}
	
	В дополнение к уже имеющимся переменным заведем ещё одну "--- номер самого старшего ненулевого бита нашего числа. Тогда $\operatorname{setZero}$ юудет просто обнулять эту переменную, $\operatorname{increment}$ проходить по элементам массива строго ДО элемента с таким индексом, а $\operatorname{get}()$ будет возвращать ноль, если указывает за старший бит. Ниже приведён псевдокод этих функций.
	
	Весьма очевидно, что они все работают за $\mathcal{O}(1)$.
	\newpage
	\begin{lstlisting}[language=Python]
	increment():
		i = 0
		while i < top_border and a[i] == 1:
			a[i] = 0
			i++
		if i < top_border:
			a[i] = 1
		if i == top_border and top_border < k:
			a[i] = 1
			top_border++
	\end{lstlisting}
	
	\begin{lstlisting}[language=Python]
	get(i):
		if i < top_border:
			return a[i]
		else:
			return 0
	\end{lstlisting}
	
	\begin{lstlisting}[language=Python]
	setZero():
		top_border = 0
	\end{lstlisting}
	
	\newpage
	
	\section{}
\label{sec:2}
	В этой задаче мы будем крайне активно использовать следующую лемму:
	\begin{lemma}
		$\forall f(n),\, g(n),\, h(n): \: f(n) = \mathcal{O}(g(n)), \, g(n) = \mathcal{O}(h(n)), \, \Rightarrow \, f(n) = \mathcal{O}(h(n))$
	\end{lemma}

	Все стрелочки в нижесприведённом списке кликабельны, по ним переносит на доказательство.

	\begin{gather*}
		1 
		\text{\hyperref[p1]{\vphantom{|}$\longrightarrow$}}
		\,
		\left(\frac{3}{2}\right)^2
		\,
		\text{\hyperref[p2]{\vphantom{|}$\longrightarrow$}}
		\,
		n^{\frac{1}{\log n}}
		\,
		\text{\hyperref[p3]{\vphantom{|}$\longrightarrow$}}
		\,
		\log\log n
		\,
		\text{\hyperref[p4]{\vphantom{|}$\longrightarrow$}}
		\,
		\sqrt{\log n}
		\,
		\text{\hyperref[p5]{\vphantom{|}$\longrightarrow$}}
		\,
		\log^2n
		\,
		\text{\hyperref[p6]{\vphantom{|}$\longrightarrow$}}
		\,
		\sqrt(2)^{\log n}
		\,
		\text{\hyperref[p7]{\vphantom{|}$\longrightarrow$}}
		\,
		2^{\log n}
		\,
		\text{\hyperref[p8]{\vphantom{|}$\longrightarrow$}}
		\,
		n
		\,
		\text{\hyperref[p9]{\vphantom{|}$\longrightarrow$}}
		\,
		n\log n
		\,
		\text{\hyperref[p10]{\vphantom{|}$\longrightarrow$}}
		\,
		\log n!
		\,
		\text{\hyperref[p11]{\vphantom{|}$\longrightarrow$}}
		\\
		n^2
		\,
		\text{\hyperref[p12]{\vphantom{|}$\longrightarrow$}}
		\,
		4^{\log n}
		\,
		\text{\hyperref[p13]{\vphantom{|}$\longrightarrow$}}
		\,
		n^3
		\,
		\text{\hyperref[p14]{\vphantom{|}$\longrightarrow$}}
		\,
		\left(\log n\right)!
		\,
		\text{\hyperref[p15]{\vphantom{|}$\longrightarrow$}}
		\,
		\left(\log n\right)^{\log n}
		\,
		\text{\hyperref[p16]{\vphantom{|}$\longrightarrow$}}
		\,
		n^{\log\log n}
		\,
		\text{\hyperref[p17]{\vphantom{|}$\longrightarrow$}}
		\,
		n\cdot 2^n
		\,
		\text{\hyperref[p18]{\vphantom{|}$\longrightarrow$}}
		\,
		\exp^n
		\,
		\text{\hyperref[p19]{\vphantom{|}$\longrightarrow$}}
		\,
		n!
		\,
		\text{\hyperref[p20]{\vphantom{|}$\longrightarrow$}}
		\,
		\left(n+1\right)!
		\,
		\text{\hyperref[p21]{\vphantom{|}$\longrightarrow$}}
		\,
		2^{2^n}
		\,
		\text{\hyperref[p22]{\vphantom{|}$\longrightarrow$}}
		\,
		2^{2^{n+1}}
	\end{gather*}
	
	
	
	\begin{proposition}
		\label{p1}
		$$\left(\frac{3}{2}\right)^2 \geq 1$$
	\end{proposition}

	\begin{proposition}
		\label{p2}
		Заменим $n = 2^k$.
		$$ n^{\frac{1}{\log(n)}} = \left(2^k\right)^{\frac{1}{k}} = 2$$
		$$ \frac{9}{8} \cdot 2 \geq \left(\frac{3}{2}\right)^2$$
	\end{proposition}

	\begin{proposition}
		\label{p3}
		$$\log(\log(n)) \geq 1$$
		\begin{center}Потому как $\lim\limits_{n \rightarrow \infty}\log(\log(n)) = \infty$.\end{center}
	\end{proposition}

	\begin{proposition}
		\label{p4}
		Заменим $n = 2^{2^k}$.
		\begin{align*}
		 \sqrt{\log n} = \sqrt{2^k} \geq \log\log n = k \Leftrightarrow\\
		 \Leftrightarrow \: 4^k \geq k^2\text{, а это верно.}
		\end{align*}
	\end{proposition}
	
	\begin{proposition}
		\label{p5}
		Заменим $n = 2^k$.
		$$\log^2n = k^2 \geq \sqrt{k} = \sqrt{\log n} $$
	\end{proposition}
	
	\begin{proposition}
		\label{p6}
		Заменим $n = 2^k$.
		$$\sqrt{2}^{\log n} = 2^k \geq k^2 = \log^2n$$
	\end{proposition}

	\begin{proposition}
		\label{p7}
		$$2^{\log n} = n \geq \sqrt{n} = \sqrt{2}^{\log n}$$
	\end{proposition}
	
	\begin{proposition}
		\label{p8}
		$$n = 2^{\log n}$$
	\end{proposition}
	
	\begin{proposition}
		\label{p9}
		$$n\log n \geq n\text{, при любом } n \geq 2$$
	\end{proposition}
	
	\begin{proposition}
		\label{p10}
		Смотреть задачу \ref{sec:3}.
	\end{proposition}
	
	\begin{proposition}
		\label{p11}
		$$n^2 \geq n\log n$$
	\end{proposition}
	
	\begin{proposition}
		\label{p12}
		$$4^{\log n} = n^2$$
	\end{proposition}

	\begin{proposition}
		\label{p13}
		$$n^3 \geq n^2$$
	\end{proposition}
	
	\begin{proposition}
		\label{p14}
		Заменим $n = 2^k$.
		$$\left(\log n\right)! = k! > 8^k$$
	\end{proposition}
	
	\begin{proposition}
		\label{p15}
		Заменим $n = 2^k$.
		$$\left(\log n\right)^{\log n} = k^k \geq k! = \left(\log n\right)!$$
	\end{proposition}
	
	\begin{proposition}
		\label{p16}
		$$n^{\log\log n} = \left(\log n\right)^{\log n}$$
	\end{proposition}
	
	\begin{proposition}
		\label{p17}
		Заменим $n = 2^k$.
		\begin{gather*}
		 2^k + k \geq k\log k \Leftrightarrow\\
		\frac{2^k+k}{k} \geq \log k \Leftrightarrow\\
		2^{\frac{2^k+k}{k}} \geq k \Leftrightarrow\\
		2^k\cdot 2^{2^k} \geq k^k \Leftrightarrow\\
		n\cdot 2^n \geq \left(\log n\right)^{\log n}
		\end{gather*}
	\end{proposition}
	
	\begin{proposition}
		\label{p18}
		\begin{gather*}
		\exp^n \geq n\cdot 2^n \Leftrightarrow\\
		\left(\frac{\exp}{2}\right)^n \geq n
		\end{gather*}
		А это верно, так как степенная функция с основанием $>1$ больше линейной с какого-то значения $n$.
		
	\end{proposition}
	
	\begin{proposition}
		\label{p19}
		Рассмотрим, например, значение $N = 27 > \exp^3$.
		Тогда $n\forall n \,\geq\, N:\; n! = 1\cdot2\cdot(3\cdot4\dotsc\cdot27) \cdot (28\cdot29\cdot\dotsc\cdot n) \,>\, 1\cdot2\cdot(\exp^{27})\cdot(\exp^{n-27}) \,>\, \exp^n$ 
	\end{proposition}
	
	\begin{proposition}
		\label{p20}
		$(n+1)! > n!$
	\end{proposition}
	
	\begin{proposition}
		\label{p21}
		Докажем по индукции:
		
		\textbf{База:} $n = 3:\; 2^{2^3} = 2^8 = 256 > 2 = (2)!$.
		
		\textbf{Переход:} заметим, что $\forall n>3:\; n! > n+1$. Таким образом, если верно, что $2^{2^n} > n!$, то $2^{2^{n+1}} = (2^{2^n})^2 > (n!)^2 > (n+1)!$. Доказано.
	\end{proposition}
	
	\begin{proposition}
		\label{p22}
		$2^{2^{n+1}} = (2^{2^n})^2 > 2^{2^n}$
	\end{proposition}

\newpage
	
	\section{}
	Идея решения такова "--- вместе с оригинальным ( $\operatorname{original}$) массивом инициализируем два массива такой же длины "---  $\operatorname{iter}$ и  $\operatorname{filled}$ и будем поддерживать количество уже инициализированных элементов  $\operatorname{cnt}$. Когда мы делаем  $\operatorname{set}$ какого-либо элемента  $\operatorname{original}$, который мы ещё не трогали, в ячейку с таким же номером в  $\operatorname{iter}$ кладём  $\operatorname{cnt}$, а в ячейку  $\operatorname{filled}$ с номером  $\operatorname{cnt}$ кладём позицию самого элемента и увеличиваем  $\operatorname{cnt}$.
	
	Теперь посмотрим, как мы будем проверять, инициализирован ли элемент на позиции  $i$. Если да, то ячейка массива $\operatorname{iter}$, на которую указывает  $\operatorname{filled}[i]$, лежит в пределе уже заполненных и должна указывать на  $i$. Почему такого не может случиться, если элемент неинициализирован, очевидно "--- первые  $\operatorname{cnt}$ элементов массива  $\operatorname{filled}$ указывают только на уже инициализированные элементы. Ниже приведён код, реализующий эту структуру данных.
	
	\lstinputlisting[language=c++]{3.cpp}\newpage
	
	\section{}
	Предлагается следующий алгоритм:
	\begin{enumerate}
		\item Разбиваем оригинальный массив на блоки по $k$ элементов;
		\item Сортируем подмассив, образованный первым и вторым блоками;
		\item Сортируем подмассив, образованный вторым и третьим блоками;
		\item $\dotsc$
		\item Сортируем подмассив, образованный предпоследним и последним блоками.
	\end{enumerate}

	Докажем, что это работает. Инвариантом будет то, что элементы в предыдущих блоках уже стоят на своих местах. Тогда если мы сортируем новую пару блоков, то все элементы, которые должны были стоять в первом блоке этой пары точно содержатся в паре, так как они не могут быть в предыдущих блоках, потому как те уже отсортированы. Значит, эти элементы встанут на свои места "--- мы отсортировали ещё один блок, отсортированный префикс увеличился, переходим дальше.
	
	Имеется $\frac{n}{k}$ блоков, каждую пару из которых мы сортируем за $\mathcal{O}(k\log k)$. Таким образом, итоговая асимптотика равна $2\cdot\frac{n}{k}\cdot\mathcal{O}(k\log k) = \mathcal{O}(\frac{n}{k}\cdot k\log k) = \mathcal{O}(n\log k)$.

\end{document}
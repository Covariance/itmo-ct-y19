\section{}
	Утверждается, что на любой перестановке, представляющей собой один цикл, сортировка выбором сделает максимальное число обменов. Покажем это по индукции:
	
	\textbf{База:} для перестановок из двух элементов очевидно.
	
	\textbf{Переход:} Так как в нашеё перестановке больше одного элемента, то первый элемент стоит \textit{не} на своём месте. Значит, на этой итерации мы совершим обмен. Поэтому, если перестановка является одним большим циклом, то на этой итерации мы совершим обмен, а оставшаяся часть перестновки останется одним большим циклом, в котором, по индукционному предположению, сортировка выбором совершает максимальное число обменов.
	
	Покажем, что на перестановках, не являющихся большими циклами, сортировка выбором совершает меньшее число обменов. От противного "--- пусть есть такая перестановка, в которой не только большой цикл, которая совершает столько же обменов. Значит, она должна совершить обмен на этой итерации и после него по предположению индукции должен остаться один большой цикл. Раз мы совершили обмен, то пройдём в элемент, который должен был стоять на месте первого и обойдём этот цикл до конца. В любом случае, этот цикл не может захватывать всю перестановку, иначе бы изначальная перестановка тоже была одним большим циклом. Значит, он захватывает не все элементы, но так как в этом цикле не меньше двух вершин, то после обмена получится не один только большой цикл, а ещё какой-то мусор. Доказано.
	
	Теперь посчитаем количество таких перестановок, что они сотавляют один большой цикл. Посмотрим на первую позицию. На неё можно поставить любой элемент, кроме первого, то есть $(n-1)$. Посмотрим, куда указывает этот элемент. На это место нельзя ставить никакой из ранее поставленных элементов, равно как и элемент, указывающий на первую позицию, то есть всего $(n-2)$, затем $(n-3)$ и так далее, до последнего элемента, который будет указывать на первую позицию. Получаем $(n-1)!$ перестановок.
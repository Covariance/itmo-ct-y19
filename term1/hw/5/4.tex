\section{}
	Будем говорить, что по компараторам единичка "спускается", а по антикомпараторам "--- "поднимается". Также во время рассуждений будем считать, что если на концах компаратора две нити с одинаковыми значаениями, то интересующая нас единица будет подниматься по антикомпаратору и спускаться по компаратору всё равно.
	
	Представим, что на каждом входе стоит единица, и "пройдём" по её пути. Каждый раз, когда мы будем встречать компаратор, по которому можно спуститься или антикомпаратор, по которому можно подняться, будем переходить по нему. Рассмотрим конкретно путь по антикомпаратору вверх. Когда встречаем такой путь на нити $i$, продолжаем следовать вверх по нему до тех пор, пока не придём на нить $j$, которая находится ниже этой. Тогда создадим компаратор $(i; j)$ в нашей сети. Сделаем так со всеми единицами, а затем выкинем все антикомпараторы. Поймём, что мы создали не более $c_2$ антикомпараторов "--- потому что для каждого антикомпаратора однозначно определён путь, по которому мы пошли, значит, однозначно определён создаваемый компаратор. Поймём, что мы создали не менее $c_2$ компараторов "--- потому что каждый антикомпаратор задействован хотя бы в одном пути, а если после прохода вверх по антикомпаратору мы не спустились обратно, то наша сеть "--- не сортирующая. Поймём, что полученная сеть "--- сортирующая. Это верно потому, что для любой единицы мы "укоротили" её путь до нужной нити.